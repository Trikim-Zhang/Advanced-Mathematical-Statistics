\section{习题2.1}
\subsection{问题}
\begin{frame}
    \frametitle{习题2.1}
    \begin{example}{习题2.1}
       设 $X_1,X_2$ 独立同分布,其共同的密度函数为 $p(x;\theta)=3x^2/\theta^3,\;0<x<\theta,\; \theta>0.$
       \begin{enumerate}
        \item 证明 $T_1=\frac{2}{3}(X_1+X_2)$ 和 $T_2=\frac{7}{6} \max(X_1,X_2)$ 都是$ \theta $的无偏估计;
    
        \item 计算 $T_1$ 和 $T_2$ 的均方误差并进行比较;
    
        \item 证明:在均方误差意义下,在形如 $T_c=c \max(X_1,X_2)$ 的估计中, $T_{8/7}$ 最优.
       \end{enumerate}
    \end{example}
\end{frame}
\subsection{第一问}
\begin{frame}[c,allowframebreaks]
    \frametitle{第一问}
    由
    \[
        \operatorname{E}(X_1)=\operatorname{E}(X_2)=\int_{0}^{\theta}x\frac{3x^2}{\theta^3}dx=\frac{1}{\theta^3}\left[ \frac{3}{4}x^4\right]^{\theta}_{0}=\frac{3}{4}\theta
    \]

    得 $\operatorname{E}(T_1)=\frac{2}{3}\operatorname{E}(X_1)+\frac{2}{3}\operatorname{E}(X_2)=\frac{2}{3} \cdot \frac{3}{4}\theta \cdot 2= \theta.$

    令$ Y=\max(X_1,X_2)$ ,因为 
    \[
        P(Y \leqslant y)=P(X_1 \leqslant y)P(X_2 \leqslant y)=P^2(X_1 \leqslant y)
    \]
    且有 
    \[
        P(X_1 \leqslant y)=\int_{0}^{y}3x^2/\theta^3dx=\frac{y^3}{\theta^3}
    \]
    故 $p_Y(y)=[P^2(X_1 \leqslant y)]'=\dfrac{6y^5}{\theta^6},$

    则 
    \[
        \operatorname{E}(Y)=\int_{0}^{\theta} y \frac{6y^5}{\theta^6} dy= \frac{1}{\theta^6}\left[ \frac{6}{7}y^7 \right]_0^{\theta}= \frac{6}{7}\theta. 
    \]
    故 $\operatorname{E}(T_2)=\frac{7}{6}\operatorname{E}(Y)=\theta . $
    证毕.
\end{frame}
\subsection{第二问}
\begin{frame}[t,allowframebreaks]
    \frametitle{第二问}
    由 
    \[
        \operatorname{E}(X_1^2)=\operatorname{E}(X_2^2)=\int_{0}^{\theta}x^2\frac{3x^2}{\theta^3}dx=\frac{1}{\theta^3}\left[ \frac{3}{5}x^5\right]^{\theta}_{0}=\frac{3}{5}\theta^2
    \]

    得 
    \[
        \operatorname{Var}(X_1)=\operatorname{Var}(X_2)=\operatorname{E}(X_1^2)-\operatorname{E}^2(X_1)=\frac{3}{5}\theta^2-\left[ \frac{3}{4}\theta \right]^2=\frac{3}{80}\theta^2
    \]
    故 
    \[
        \operatorname{Var}(T_1)=\frac{4}{9}\operatorname{Var}(X_1)+\frac{4}{9}\operatorname{Var}(X_2)=\frac{4}{9} \cdot \frac{3}{80}\theta^2 \cdot 2=\frac{1}{30}\theta^2.
    \]

    由 
    \[
        \operatorname{E}(Y^2)=\int_{0}^{\theta}y^2\frac{6y^5}{\theta^6}dy=\frac{1}{\theta^6}\left[ \frac{6}{8}y^8\right]^{\theta}_{0}=\frac{3}{4}\theta^2
    \]

    得 
    \[
        \operatorname{Var}(Y)=\operatorname{E}(Y^2)-\operatorname{E}^2(Y)=\frac{3}{4}\theta^2-\left[ \frac{6}{7}\theta \right]^2=\frac{3}{4 \cdot 49}\theta^2,
    \]

    故$ \operatorname{Var}(T_2)=\frac{49}{36}\operatorname{Var}(Y)=\frac{1}{48}\theta^2.$

    故有 
    \[
        \operatorname{MSE}(T_1)=\operatorname{Var}(T_1)=\frac{1}{30}\theta^2>\frac{1}{48}\theta^2=\operatorname{Var}(T_2)=\operatorname{MSE}(T_2).
    \]
\end{frame}
\subsection{第三问}
\begin{frame}[t,allowframebreaks]
    \frametitle{第三问}
    由 $\operatorname{E}(T_c)=c\operatorname{E}(Y)$,有

    \[
        \begin{aligned} \operatorname{MSE}(T_c)&=\operatorname{E}(T_c-\theta)^2=\operatorname{Var}(T_c)+\operatorname{E}^2(T_c-\theta) \\ &=c^2\operatorname{Var}(Y)+[c\operatorname{E}(Y)-\theta]^2 \\ &=c^2\frac{3}{4 \cdot 49}\theta^2 +[c \frac{6}{7}\theta-\theta]^2 \\ &=\left[ \frac{3}{4 \cdot 49}c^2+ \left( \frac{6}{7}c -1 \right)^2\right] \theta^2 \\ &=\left[ \frac{3}{4}c^2-\frac{12}{7}c+1 \right] \theta^2, \end{aligned} 
    \]
     故当 $c=-\dfrac{-\frac{12}{7}}{2\cdot\frac{3}{4}}=\frac{8}{7}$ 时,上述 $\operatorname{MSE}(T_c)$ 取得最小值 $\dfrac{1}{49}\theta^2 $. 证毕.
\end{frame}