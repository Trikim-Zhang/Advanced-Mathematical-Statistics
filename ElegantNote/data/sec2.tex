\section{点估计}
\subsection{基本概念}
\begin{definition}[点估计]
    设参数$\theta$,$X$为样本,用统计量$\hat{\theta}(X)$作为未知参数$\theta$的“猜测”称为点估计。
\end{definition}
该估计的均方误差MSE为
\[
    \begin{array}{ll}
        \operatorname{MSE}(\hat{\theta}) &= \mathrm{E}(\hat{\theta}(X)-\theta)^2=\mathrm{E}\{[\hat{\theta}(X)-\mathrm{E}(\hat{\theta}(X))]-[\theta-\mathrm{E}(\hat{\theta}(X))]\}^2\\
        &=\mathrm{E}[\hat{\theta}(X)-\mathrm{E}(\hat{\theta}(X))]^2 + \mathrm{E}[\theta-\mathrm{E}(\hat{\theta}(X))]^2\\
        &=\operatorname{var}(\hat{\theta}) + \operatorname{bias}(\hat{\theta})
    \end{array}
\]
\subsection{无偏估计及UMVUE}
\subsubsection{无偏估计}
\begin{definition}[无偏估计]
    设$(\mathscr{X},\mathscr{B},\{P_{\theta},\theta\in\Theta\})$为可控参数统计结构,$g(\theta)$是未知参数,$X = (X_1,\cdots,X_n)$是来自该统计结构的一个样本,若用$\hat{g}(X)$估计$g(\theta)$,且
    \[
        \operatorname{E}_{\theta}(\hat{g}(X)) = g(\theta),\forall \theta\in\Theta
    \]
    则称$\hat{g}(X)$为$g(\theta)$的无偏估计。
\end{definition}
\begin{remark}
    关于无偏估计,我们有三点需要说明
    \begin{enumerate}
        \item 无偏估计不一定存在
        \item 对可估参数,无偏估计一般不唯一
        \item 无偏估计并不一定是一个好估计
    \end{enumerate}
\end{remark}

\subsubsection{UMVUE}
\begin{definition}
    设$g(\theta)$是可估参数,$T(X)\in U_g$,若对于$\forall \phi(X)\in U_g$有
    \[
        \operatorname{var}_{\theta}T(X)\leq \operatorname{var}_{\theta}\phi(X),\forall \theta\in\Theta
    \]
    则称$T(X)$是$g(\theta)$的一致最小方差无偏估计(UMVUE)。
\end{definition}
\begin{lemma}
    设$S(X)$是分布族$\left\{ p_{\theta},\theta\in\Theta \right\}$的充分统计量,$\phi(X)$是$g(\theta)$的无偏估计,令$T(X) = \operatorname{E}(\phi(X)|S(X))$,则$T(X)$也是$g(\theta)$的无偏估计,且$\operatorname{var}T(X)\leq \operatorname{\phi}(X)$
\end{lemma}
\begin{remark}
    求UMVUE的方法
    \begin{enumerate}
        \item 寻找充分完备统计量的函数$S(X)$,使其属于$U_g$
        \item 任取$\hat{g}(X)\in U_g$,令$T(X) = \operatorname{E}(\hat{g}(X)|S(X))$
    \end{enumerate}
\end{remark}
\begin{example}
    设$X = (X_1,\cdot,X_n)$是来自$b(1,\theta),\theta\in(0,1)$的一个样本,由指数组性质知,$S(X)=\sum X_i$是充分完备统计量。
    \begin{enumerate}
        \item 对$\theta$,因为$E(S(X)) = n\theta$,则$\bar{X} = S(X)/n$是$\theta$的无偏估计,从而也$\theta$的UMVUE。
        \item 对于$g(\theta) = \theta^k + (1-\theta)^{n-k},0\leq k\leq n$,如何找$g(\theta)$的UMVUE?
    \end{enumerate}
    对于$g(\theta) = \theta^k + (1-\theta)^{n-k},0\leq k\leq n$,先找一个$g(\theta)$的无偏估计$\phi(X)$。令
    \[
        \phi_1(X)=\begin{cases}1,&\sum\limits_{i=1}^kX_i=k\\0,&others\end{cases},\quad
        \phi_2(X)=\begin{cases}1,&\sum\limits_{i=k+1}^nX_i=0\\0,&others\end{cases}
    \]
    又令$\phi(X) = \phi_1(X) + \phi_2(X)$,则
    \[
        \operatorname{E}(\phi(X)) = \operatorname{E}(\phi_1(X))+\operatorname{E}(\phi_2(X)) = \theta^k + (1-\theta)^{n-k} = g(\theta)
    \]
    所以$\phi(X)\in U_g$。故而$T(X) = \operatorname{E}(\phi(X)|S(X))$是$g(\theta)$的UMVUE。
    当$k\leq s$时,
    \[
        \begin{array}{l}
            \operatorname{E}(\phi_1(X)|S(X)) = P(\phi_1(X) = 1|S(X) = s)\\
            =\dfrac{P(\phi_1(X) = 1,S(X) = s)}{P(S(X) = s)} = \dfrac{P(\sum\limits_{i=1}^kX_i=k,\sum\limits_{i = k+1}^{n}X_i = s-k)}{P(\sum\limits_{i = 1}^{n}X_i = s)}\\
            =\dfrac{\theta^k\cdot C_{n-k}^{s-k}\theta^{s-k}(1-\theta)^{n-s}}{C_{n}^{s}\theta^k(1-\theta)^{n-k}} = \dfrac{C_{n-k}^{s-k}}{C_{n}^{s}} 
        \end{array}   
    \]
    当$k\geq s$,$\operatorname{E}(\phi_1(X)|S(X)) = 0$。

    同理,当$k\geq s$时,$\operatorname{E}(\phi_2(X)|S(X)) = \dfrac{C_{k}^{s}}{C_{n}^{s}}$
    
    当$k < s$时,$\operatorname{E}(\phi_2(X)|S(X)) = 0$。

    因此,$g(\theta)$的UMVUE为
    \[
        T(X) = \left\{
            \begin{array}{ll}
                \dfrac{C_{n-k}^{s-k}}{C_{n}^{s}} & k\leq s\\
                \dfrac{C_{k}^{s}}{C_{n}^{s}} & k>s
            \end{array}
        \right.
    \]
\end{example}
\begin{example}
    设$X_1,\cdots,X_n$是来自总体$X$的几何分布,其概率分布为$P_{\theta}(X = x) = (1-\theta)^x\theta,x = 0,1,\cdots,$。试求$\theta$的UMVUE。

    对于$\phi(X)$
    \[
        \phi(X) = \left\{
            \begin{array}{ll}
                1, & X_{1} = 0 \\
                0, & others
            \end{array}
        \right.
    \]
    有$\operatorname{E}(\phi(X)) = P_{\theta}(X_1 = 1) = (1-\theta)^0\theta = \theta$。且$\sum_{i = 1}^{n}X_i\sim Nb(n;s,\theta)$服从负二项分布。
    因此,
    \[
        \begin{array}{l}
            \operatorname{E}(\phi(X)|S(X)) = P(\phi(X) = 1|S(X) = s)\\
            =\dfrac{P(\phi_1(X) = 1,S(X) = s)}{P(S(X) = s)} = \dfrac{P(X_1=0,\sum\limits_{i = 2}^{n}X_i = s)}{P(\sum\limits_{i = 1}^{n}X_i = s)}\\
            =\dfrac{\theta C_{s+n-2}^{n-2}\theta^{n-2}(1-\theta)^{s}}{C_{s+n-1}^{n-1}\theta^{n-1}(1-p)^{s}} = \dfrac{n-1}{s+n-1}\\
        \end{array}   
    \]
\end{example}
\begin{example}
    设$X_1,\cdot,X_n$独立同分布,均服从区间$(0,\theta),\theta>0$上的均匀分布,样本为$X = \left( X_1,\cdots,X_n \right)$。求参数$g(\theta) = \theta^2$的UMVUE。

    首先,我们知道$X_{(n)}$是为一个完全充分统计量。

    \textbf{直接方法}:找到一个合适的$X_{(n)}$的函数$\phi(X_{(n)})$,使得$\phi(X_{n})$称为$g(\theta) = \theta^2$的无偏估计,即$E_{\theta}(\phi(X_{n})) = \theta^2$。为此,首先注意到$X_{(n)}$的概率密度函数为
    \[
        p(t;\theta) = \left\{
            \begin{array}{ll}
                n\dfrac{t^{n-1}}{\theta^n}, & 0<t<\theta\\
                0, others
            \end{array}
        \right.
    \]
    我们看一下其期望
    \[
        E_{\theta}X_{(n)} =  n\int_{0}^{\theta}n\dfrac{t^{n-1}}{\theta^n}t  \,dx = \dfrac{n}{n+1}\theta
    \]
    \[
        E_{\theta}X_{(n)}^2 =  n\int_{0}^{\theta}n\dfrac{t^{n-1}}{\theta^n}t^2  \,dt = \dfrac{n}{n+2}\theta^2
    \]
    于是我们就有$E_{\theta} \left( \dfrac{n+2}{n}X_{(n)}^2 \right) = \theta^2$就得到$ \dfrac{n+2}{n}X_{(n)}^2 $就是$\theta^2$的无偏估计,进而是UMVUE。

    \textbf{条件期望法}:我们先找到$\theta^2$的一个无偏估计,$E_{\theta}X_{1}^2 =\int_{0}^{\theta}t^2\frac{1}{\theta} \,dt= \frac{\theta^2}{3}$,课件$3X_{1}^2$是$\theta^2$的一个无偏估计,进而求期望$E_{\theta}\left[ 3X_{1}^2|X_{(n)} \right]$.
    \[
        \begin{array}{ll}
            E_{\theta}\left[ 3X_{1}^2|X_{(n)} =t \right]&=\dfrac{1}{n}
            \cdot 3t^2 + (1-\dfrac{1}{n})\int_{0}^{t}3u^2\dfrac{1}{t}\, du\\
            &=\dfrac{3t^2}{n} + \dfrac{n-1}{n}t^2\\
            &=\dfrac{n+2}{n}t^2
        \end{array}
    \]
    可见,$\dfrac{n+2}{n}X_{(n)}^2$为$\theta^2$的UMVUE
\end{example}

\subsection{极大似然估计(MLE)}
极大似然估计在直观上可以这样解释:使得出现所选样本最大概率的分布参数的估计。
\begin{definition}
    设$X\sim f(x;\theta),\theta\in\Theta$,把$f(x;\theta)$是为$\theta$的哈桑农户,则称它为$X$关于$\theta$的似然函数,$L(\theta,X) = \ln f(x;\theta) = L(\theta)$称为对数似然函数,若$\hat{\theta}$满足
    \[
        f(x,\hat{\theta}(x)) = \max_{\theta\in\Theta}f(x;\theta)
    \]
    称$\hat{\theta}$为$\theta$的极大似然估计。
\end{definition}
\begin{example}
    设$X_1,\cdots,X_n$是来自$b(1,\theta)$的一个样本,$0<\theta<1$
    \[
        l(\theta;x) = \left( \sum x_i \right)\ln \theta + \left( n - \sum x_i \right)\ln (1-\theta) 
    \]
    由$\dfrac{\partial l}{\partial \theta} = 0$知
    \[
        \hat{\theta} = \bar{x}
    \]
\end{example}
\begin{remark}
    MLE还有一个非常有吸引力的性质:如果$\hat{\theta}$是$\theta$的MLE,$g(\cdot)$为可测函数,那么$g(\hat{\theta})$也是$g(\theta)$的MLE。

    对于多参数指数族$p(x;\theta) = \exp\{ \sum_{j = 1}^{k}\theta_jT_j(x)+c(\theta)+d(x) \}$,似然方程化为
    \[
        \sum T_{j}(x_i) = -n\dfrac{\partial c(\theta)}{\partial \theta_j}
    \]
\end{remark}
\begin{theorem}[MLE的相合性]
    设$\{ p_{\theta}(x;\theta):\theta\in\Theta \}$是可识别的,且$p_{\theta}(x;\theta)$关于$\theta$可微,则似然方程在$n\to \infty$时,以概率1有解,且此解关于$\theta$时相合的。
\end{theorem}
\begin{theorem}[MLE的渐进正态性]
    在一系列正则条件下,对于$p(x;\theta)$的相合解$\hat{\theta}_{n}$,有
    \[
        \sqrt{n}(\hat{\theta}_n-\theta_0)\overset{L}{\to}N(0,I^{-1}(\theta_0))
    \]
    其中$\theta_0$为真值。
\end{theorem}